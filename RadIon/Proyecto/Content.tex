\begin{abstract}
    Los rayos $x$, o radiación $x$ es una forma de radiación electromagnética penetrante de alta energía con aplicaciones médicas importantes. En este texto se discutirá la física detrás de la generación de rayos $x$ y la calidad del haz, es decir la habilidad de penetración de este.
\end{abstract}


\section{Producción de rayos x y espectro de energía}
%Whenever charged particles (electrons or ions) of sufficient energy hit a material, X-rays are produced.
``Cuando una partícula cargada (iones o electrones) con suficiente energía golpea un material, son producidos rayos $x$.''
\cite{XrayWiki44:online}

\subsection{Rayos x por fluorescencia}
%The term fluorescence is applied to phenomena in which the absorption of radiation of a specific energy results in the re-emission of radiation of a different energy (generally lower).
``El término fluorescencia es aplicado al fenómeno en el que la absorción de radiación a una energía específica, resulta en la re-emisión de radiación con un valor de energía distinto (usualmente menor).''
\cite{Xrayfluo48:online}

% \biskip

Los rayos $x$ son producidos por fluorescencia cuando ocurren fuertes colisiones entre partículas cargadas y electrones de la capa interna de un material, cuando la vacante creada se llena de nuevo se genera fluorescencia de rayos $x$. Aunque solamente una pequeña fracción (menor al $1\%$) de la energía de las partículas cargadas utilizada en las interacciones de colisión termina en fluorescencia de rayos $x$.
\linebreak
\cite{attix2008introduction}

\subsubsection{Rendimiento de fluorescencia}
Se le llama rendimiento de fluorescencia a la probabilidad de un rayo $x$ producido por fluorescencia, logre escapar del átomo que lo originó.

El rendimiento de fluorescencia se distingue como $Y_K$ para una vacancia de una capa $K$ y de la misma forma para el resto de capas.

\subsubsection{Distribuciones direccionales de la fluorescencia}
Ya que la fluorescencia es emitida a partir de un segundo proceso de transmisión que resulta de un primer evento de ionización, no existe correlación angular alguna entre la dirección de la partícula incidente y el fotón emitido por fluorescencia.
Por tanto, la fluorescencia es emitida de forma isotrópica con respecto a la energía y la intensidad, despreciando la atenuación de rayos que escapan del objetivo.
\linebreak
\cite{attix2008introduction}

\subsection{Rayos x por Bremsstrahlung}

``Estrictamente hablando, se entiende por radiación de frenado [o Bremsstrahlung] a cualquier radiación debida a la aceleración de una partícula cargada, como podría ser la radiación de sincrotrón; pero se suele usar sólo para la radiación de electrones que se frenan en la materia.''
\cite{Radiaci:online}


\subsubsection{Eficiencia de producción}
La generación de rayos $x$ por Bremsstrahlung se hace acelerando un haz de electrones para que choque contra un objetivo metálico. Los objetivos con un alto número atómico alto convierten una fracción mayor de la energía en rayos x producidos por Bremsstrahlung que los objetivos con bajo número atómico.
El tungsteno es una elección común, ya que tiene un número atómico alto pero también un punto de fusión, ya que la energía que no es empleada en producción de rayos $x$ por efecto Bremsstrahlung termina produciendo ionización y excitación por interacciones de colisión; esta energía degrada en calor en el objetivo (con excepción de la pequeña parte de emisión de rayos x producidos por fluorescencia) y es por esto que se requiere un mecanismo de enfriamiento.

\section{Filtración de rayos x y calidad del haz}
Como su nombre lo indica, la filtración de rayos $x$ es una técnica para filtrar por longitudes de onda para algún haz dado.

Y como fue mencionado antes, la calidad del haz nos habla sobre la habilidad de penetración de este.


\subsection{Filtración de rayos x}
La filtración de rayos $x$ consiste en colocar un material en frente de una fuente de rayos $x$ para reducir la intensidad de longitudes de onda particulares del espectro de emisión y selectivamente alterar la distribución de las longitudes de onda de los rayos $x$ para un haz dado.

\cite{Xrayfilt39:online}

Podemos clasificar la filtración a partir de la energía cinética $T_o$ de los rayos x:

\subsubsection{Filtración para $\mathbf{T_o\lesssim 300 ~ keV}$}

El resultado principal de agregar este tipo de filtros a un haz de rayos $x$ es remover fotones preferencialmente a energías donde el coeficiente de atenuación es mayor. El efecto fotoeléctrico es la interacción dominante en fotones por debajo de los cientos de $keV$,
y como el coeficiente de interacción fotoeléctrica varía aproximadamente como $(1/h\nu)^3$ en este rango de energías, entonces la parte más baja del espectro parece como si gradualmente fuera reducida conforme se agrega más y más filtración.

Los materiales más comunes para la filtración de rayos $x$ son el cobre, estaño, plomo y aluminio, y estos son utilizados en combinación o solos. La ventaja de combinarlos es que generalmente son más capaces de hacer más angosto el espectro a un grado deseado mientras se preserva más de la salida los rayos $x$ de lo que se preserva utilizando un solo material.

\cite{attix2008introduction}

\subsubsection{Filtración para $\mathbf{T_o\gtrsim 300 ~ keV}$}

A energías más altas, el efecto fotoeléctrico se vuelve despreciable al lado del las interacciones por efecto Compton, y el coeficiente total es menos dependiente de la energía, por lo que la filtración del espectro de rayos $x$ generados por electrones con energías en el orden de los megavolts, principalmente remueve los fotones por debajo de los cientos de $keV$ sin gran modificación a la forma del espectro a altas energías.

Aun así, el uso de un filtro más grueso como el plomo, tiende a filtrar los fotones de más alta energía que se generan por producción de pares, así como los de más baja energía que se generan por efecto fotoeléctrico.

\cite{attix2008introduction}

\subsection{Especificación de la calidad del haz de rayos x}

La calidad de un haz de rayos $x$ puede ser especificada en términos del espectro, pero también a partir de las características de su atenuación en un medio de referencia. Aunque hay que tener en cuenta que es posible que el haz no tenga una calidad uniforme en toda el área transversal.

\cite{attix2008introduction}


\subsubsection{Estimación del espectro}

El espectro de emisión puede ser estimado desde consideraciones teóricas, o derivado de la transformada de Laplace de la curva de atenuación medida de un haz angosto. Aunque este segundo método es relativamente más difícil y provee una mejora poco considerable.

\cite{attix2008introduction}
% A altas energías, grandes sentelladores 

\subsubsection{Estimación de las curvas de atenuación para energías mayores de $300keV$}

Es posible (con algunas limitaciones), derivar el espectro a partir de la forma de la curva de atenuación.
Quiere decir que la forma de esta curva debería ser una especie de firma para su respectivo espectro. Esto es, cada espectro de haz de rayos $x$ está únicamente relacionado con la forma de una única curva de atenuación en un medio dado. Por lo que los datos de atenuación pueden ser utilizados para caracterizar un haz de rayos $x$.

Para generalizar estos datos, se hace la siguiente convención:

\begin{itemize}
    \item Se usa aluminio puro o cobre como medio de atenuación.
    \item Se requiere de una geometría de haz angosto (es decir, los rayos del atenuador no deben llegar al detector).
    \item El detector debe ser equivalente al aire, es decir, debe dar una respuesta constante por unidad de exposición, independiente de la energía del fotón.
\end{itemize}


\cite{attix2008introduction}

\nocite{*}