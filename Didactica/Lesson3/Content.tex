Tal y como fue visto en esta lección, la atención se puede definir como la capacidad de seleccionar y concentrarse en ciertos estímulos que sean relevantes \cite{AtenCognifit}.

Durante cualquier clase, estímulos de distracción abundan: lo que  ocurre fuera del salón, la broma que está haciendo tu compañero de clase, lo que se escucha del salón adyacente, el bigote del profesor y muchas cosas más. 

Pero en una clase en línea, la cantidad de estímulos de distracción es aún mayor. Tienes acceso y libertad de hacer lo que quieras en tu casa, mientras apareces "conectado" en la llamada. Concentrarse en cualquier clase en línea es algo bastante difícil de lograr. 

Pensemos ahora que ocurre usualmente en una clase de física: El profesor llega, saluda y comienza a dar el tema del día. Dependiendo del profesor este tema puede abarcar una porción pequeña, decente o sobrecargada de contenido para una clase de 2 horas. El profesor suele pasar hablando sin mucha pausa y con ningún cambio de actividad durante las 2 horas que su período dura, se le acaba el tiempo y dice "seguimos la otra clase" y termina. Terrible, poner atención en una clase en línea de por sí difícil, que la clase sea de esta manera no suele ayudar mucho.

Vemos tres categorías importantes, relacionadas con la atención de los estudiantes durante un curso de física en línea: los factores de distracción en sí, la forma del profesor de impartir la clase y lo que el alumno hace o no hace para mantener la atención. Revisemos estos de forma detallada a continuación:

\section{Factores de distracción}

Ya hablamos de como la libertad de aparecer "en línea" en la clase mientras haces otra cosa hace difícil la concentración de los estudiantes. Examinemos los factores de distracción más comunes debidos a este hecho:

\subsection{Redes sociales}
Las redes sociales cada vez están más presentes y son más normalizadas en la vida de las personas. Y mientras en las clases presenciales eran prohibidos o al menos mal vistos, el uso de teléfonos por ejemplo; en línea los estudiantes tienen, no solo acceso a todos sus dispositivos electrónicos e internet, sino que esto es un requisito, porque de lo contrario ¿Cómo recibirías la clase?

Además, con la falta de interacción social física que sufrimos estos días, las redes sociales se vuelven el único medio de interacción con otras personas. Esta interacción social es una necesidad humana: La necesidad emocional. Por supervivencia, el hombre necesita de sentir que pertenece, y necesita estar ligado a una familia o un grupo de amigos \cite{EmotionalNeeds}.

\subsection{Actividades recreativas}
Todos disfrutamos de realizar actividades recreativas, desde jugar con nuestros amigos, salir a correr, leer un libro, hasta cocinar o ir de viaje. Además, este tipo de actividades son cruciales para una buena salud física y mental en los individuos \cite{Recre1}. Realizar estas actividades tiene otra serie de beneficios, como sacarte de la rutina, y darte perspectivas distintas de la vida \cite{Recre2}.

El problema es que, este tipo de actividad no debe realizarse a la mitad de una clase de física. Quitando no solo la atención de la clase, sino también de la actividad recreativa en sí, quitando probablemente los beneficios que esta conlleva. Hace algunos años, nadie hubiera imaginado que las personas se irían de vacaciones a la mitad de la universidad, ahora es común ver que la gente está recibiendo sus clases mientras suben una montaña rusa del otro lado del mundo. No digamos actividades como jugar en la misma computadora en la que estás recibiendo la clase.

\subsection{Tareas y otros}
Otra distracción debido a la libertad que se tiene en casa, es que el estudiante puede hacer las tareas que tiene pendientes, ya sea de ese curso o de otro, o incluso de otra cosa que poco tenga que ver con la carrera, como limpiar su cuarto o doblar su ropa. 

Estas actividades solían realizarse en un horario distinto al de clases, porque ¿cómo podrías limpiar tu cuarto si estás en la universidad?

\subsection{Otros factores}
Gracias a la virtualidad, se han introducido muchos factores de distracción nuevos y no todos los hemos abordado. Pero aparte de estos, hay que tomar en cuenta que los factores antiguos aún están presentes. Alguien puede ser muy responsable y conectarse a la clase y no hacer más que tratar de poner atención durante las 2 horas, pero si algo extraño pasa por su ventana, o si alguien le habla, igual será distraído aunque no sea su intención.

\section{El profesor}
Hablemos ahora sobre la influencia del profesor en la atención del estudiante. 

Como ya vimos, debido a la virtualidad, la cantidad de distracciones se incrementó agresivamente. Por esta razón, el papel del profesor en retener la atención de sus estudiantes también se ve aumentado. Si el profesor no captura la atención de sus estudiantes, será mucho más fácil que se vean tentados a dirigir su atención a una actividad más estimulante.

\subsection{El método de enseñanza}
El método de enseñanza es uno de los factores más importantes que mantienen la atención de los estudiantes. Un estudiante que sabe que en cualquier momento pueden preguntarle como hacer un ejemplo, estará mucho más pendiente de la clase que uno que sabe que el profesor solo hablará por 2 horas.

Un problema grandísimo es que heredó el método de enseñanza presencial a una enseñanza en línea. El método siempre debe adaptarse al entorno y al grupo de estudiantes, forzar a estudiantes a adaptarse a un método que no funciona para ellos solo logrará la pérdida de interés; y obviamente no se puede forzar al entorno a cambiar para ajustarse al método.

Una propuesta de método adecuado para esta situación, fue planteada por un estudiante de física en Guatemala: ``Eliminar el sistema de clase magistral, reforzar el aprendizaje autodidacta enviando lecturas, videos y otras referencias de aprendizaje al estudiante, y aprovechar el tiempo de clase para la resolución de dudas, ejemplos, aplicaciones y experimentos.''

Pero como esperar este cambio tan drástico en la enseñanza de muchos profesores es poco probable, una propuesta alternativa es mejorar las clases magistrales, cambiando algunas cosas como: realizar pequeñas pausas para que el estudiante pueda digerir el contenido, calcular el contenido impartido para que no sobresature al estudiante, pero que tampoco lo aburra con demasiadas repeticiones (las cuales son un factor de distracción también \cite{Lec5}), realizar al menos un receso en períodos de más de 50 minutos y guardar al menos 10 minutos para realizar un pequeño resumen a forma de conclusión de lo aprendido ese día. 

\subsection{La programación de la clase}
No hay duda que la mayoría de los profesores tiene una buena formación y excelente conocimiento en su área. El problema es que no siempre pueden compartirlo de la forma más óptima. Ocasiones en las que hay una confusión en clase pueden ser evitadas con la programación diaria del curso, aunque esto claro, será un tiempo extra que los profesores deben de consumir, y por tanto, de no estar considerado en el salario de estos, es algo que las universidades deben de tomar en cuenta, remunerar y auditar. 

\section{Los estudiantes}

Finalmente llegamos a los estudiantes, quienes deben poner de su parte y no esperar que todo les llegue masticado a la boca.

Cada estudiante es distinto, y por esto, es importante que cada uno conozca bien su propia forma de optimizar su aprendizaje y mejorar su atención. Existen una infinidad de recursos que nos ayudan a determinar esto, y se han tratado de clasificar los tipos de aprendizaje (por ejemplo \cite{LearnStyle}) para las personas.

Sin embargo, también es posible (y probablemente ideal) determinar esto con base en la experiencia y la experimentación. Cada estudiante puede intentar distintos métodos hasta encontrar alguno que se adapte a ellos.

Además de esto, otra cosa que ayuda a mantener la atención de los estudiantes en esta modalidad, es proponerse y forzarse a participar y preguntar al menos una $n$ cantidad de veces. De este modo, aún que el maestro no pregunte a sus estudiantes, el estudiante se forzará a prestar atención para poder hacer preguntas congruentes y participaciones que aporten al tema.