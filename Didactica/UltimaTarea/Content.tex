\section{Competencias}

\subsection{Competencias generales}

De las competencias originales, estoy muy de acuerdo con una en particular:
``\textbf{Demostrar hábitos de trabajo necesarios para el desarrollo de la profesión tales como el trabajo en
equipo, el rigor científico, el auto-aprendizaje y la persistencia.}''

A diferencia de las otras, me parece que esta está planteada correctamente como una competencia y no como un objetivo de clase o meta.

Además de esta, yo agregaría una competencia que hable sobre el desarrollo de la forma de pensar de un físico, aunque no estoy muy seguro de como redactarlo.

Las competencias que personalmente incluiría en mi programa serían:

% Desarrollar la capacidad de abstraer los problemas de física.
\begin{itemize}
    \item Desarrollar la capacidad de comprensión y resolución de problemas.
    \item Activar el tipo de razonamiento adecuado para la profesión de investigador científico.
    \item Desarrollar hábitos de trabajo necesarios para ejercer la profesión tales como el trabajo en equipo, el rigor científico, el autoaprendizaje y la persistencia.
\end{itemize}

\subsection{Competencias específicas}

Es difícil lograr colocar este tipo de competencias sin acercarse al concepto de objetivo, sin embargo, en mi intento de lograrlo, las unificaría y las modificaría un poco:

\begin{itemize}
    \item Internalizar los conceptos de velocidad, fuerza, cantidad de movimiento, trabajo, energía y las leyes fundamentales de la Dinámica Newtoniana. De tal modo sean visualizados por el estudiante de forma automática en cualquier proceso físico al que se enfrente, y no solamente en el ámbito académico.
    \item Desarrollar una intuición física.
\end{itemize}

Otra opinión que me gustaría agregar sobre el programa original, es que vemos que está presente una intención de desarrollar un hábito como el trabajo en equipo, sin embargo no está planificada ninguna actividad de zona que explícitamente indique que será realizada en equipo.

\section{Evaluación del curso}

\subsection{Competencias generales en mente:}\label{sec:gen}
Para desarrollar las competencias generales, fomentaría la participación en clase, además de transmitirlas por con el ejemplo. Estas también suelen desarrollarse bien con la resolución de exámenes escritos que exploten el razonamiento de los estudiantes; lo que suele conseguirse con problemas que utilicen los conceptos aprendidos, pero hacen necesaria una comprensión más profunda de estos. 
Finalmente los proyectos se alinean muy bien con estas competencias, abarcado aquí una específica que (aparte de en clase) no se había desarrollado, y es el trabajo en equipo.

\subsection{Competencia específica en mente:}\label{sec:esp}
Lo propuesto anteriormente se ajusta de muy buena forma a las competencias específicas a desarrollar. Creo que para lograr internalizar los conceptos físicos, es necesaria una interacción directa con los fenómenos, y esto también contribuye enormemente a desarrollar una intuición física. Por lo que en definitiva agregaría una sección experimental al curso a pesar de que este es uno de física teórica. 

Esto es fácil de implementar, ya que dentro de los proyectos se incluiría una parte de desarrollo experimental cuyo procedimiento podría ser realizado en una clase para tener al profesor como apoyo. Para estos temas tan básicos, los experimentos suelen ser muy simples, pero lograr entender como se relaciona la teoría con estos fenómenos facilita mucho más la internalización de los conceptos.
Incluso, realizar experimentos mentales, que involucren únicamente la imaginación del estudiante es una alternativa también, ya que, como dije anteriormente, la teoría es muy básica.

En línea con el resto de actividades, claramente los exámenes serían sobre los temas desarrollados en clase, que a su vez son elegidos en línea con las competencias específicas. La participación en clase también iría directamente en esta línea, y aquí se puede incluir la parte de los experimentos mentales mientras se desarrolla la clase.

\subsection{Evaluación del curso:}

\begin{table}[H]
    \centering
    \begin{tabular}{|lr|}
        \hline
        Participación en clase & 5pts\\
        2 Proyectos & 30pts\\
        4 Exámenes parciales & 40pts\\
        &\\
        Zona & 75pts\\
        Final &25pts\\
        \hline\hline
        Nota & 100pts\\
        \hline
    \end{tabular}
\end{table}


\subsection{Explicación}
En las secciones \ref{sec:gen} y \ref{sec:esp} se describe el porqué de la selección de estas actividades, sin embargo aún hace falta detallar el porqué de las cantidades y ponderaciones de cada una.

\subsubsection{Participación en clase}
La idea es incentivar a los estudiantes a participar, pero no que ganen la clase por participación, por lo que el punteo por esto es bajo. Sin embargo, conforme el curso avance yo esperaría que los estudiantes descubran que no solo necesitarán esos 5 puntos más adelante, sino que necesitarán el aprendizaje que les da la participación en clase y la retroalimentación directa del catedrático.

\subsubsection{Proyectos}
La cantidad de proyectos se basó en los temas a desarrollar en el curso, la idea es que uno de los proyectos les sirva para internalizar toda el área de cintemática y dinámica (movimiento y Leyes de Newton) y otro proyecto específicamente para el área de energía (trabajo, energía y momentum).

\subsection{Exámenes parciales}
La idea de hacer tantos parciales, es que estos sean cortos, con una sección de preguntas directas de comprensión del tema, y un solo problema que asocie todos los conceptos vistos en esa sección. De tal modo que los estudiantes concentren todo su esfuerzo en desarrollar una sola idea de un problema complejo (esto en línea con la mayoría de competencias generales y específicas).

El punteo de cada parcial no es tan alto, pero en conjunto si lo son. Probablemente no lo logren una nota excelente con el primero, pero tendrán 3 oportunidades más para levantar su nota, de este modo, se consigue que los estudiantes realmente se esfuercen en comprender a fondo los temas.

\subsection{Ejercicios propuestos}
Dentro de la zona no se incluye ninguna sección de tareas, sin embargo, proponer ejercicios para el estudio de los temas es una forma eficiente de lograr que los estudiantes realicen los ejercicios de forma consiente (ya que estos no serán recompensados con ninguna nota por ello, pero la resolución de estos en definitiva será de ayuda en sus exámenes parciales).

\subsection{Examen final}
Yo realizaría un examen final con un formato parecido pero extendido a los parciales. De nuevo con una sección de preguntas, posiblemente repetidas de los parciales, esperando ver una mejora en el caso de los estudiantes que fallaron la primera vez. Y una sección de problemas, esta vez 2 o 3, con al rededor de la misma dificultad que los parciales, pero que sean problemas que necesiten contenidos mezclados, por ejemplo: un sistema que se resuelva con energías, pero que termine con un tiro parabólico que se resuelve con cinemática.
Suena complicado, pero estos temas se prestan mucho a esto, y en libros ya hay algunos problemas que tienen este tipo de problemas ``acumulativos''.

\subsection{Mi postura ante estas evaluaciones}
Creo que, con estos exámenes se estaría evaluando el dominio de los temas, específicamente las preguntas pueden decir mucho sobre la comprensión a fondo de un tema, aunque por alguna razón el estudiante no logre resolver el problema.

Aparte de esto, creo que el programa está diseñado para que solo un estudiante que haya alcanzado (en cierto nivel) las competencias propuestas, logre ganar el curso. Pero también da la posibilidad a estudiantes a ganar con puro esfuerzo aunque sus habilidades no sean tan altas.


\section{Ejemplo de Examen Parcial}

\subsection*{Serie 1 (4pts)}
\textbf{Instrucciones: Conteste las siguientes preguntas y justifique brevemente su respuesta.}

\begin{enumerate}
    \item ¿Puede un cuerpo estar en equilibrio, si una única fuerza actúa sobre él?
    \item ¿Puede la fuerza normal ser distinta del peso? De un ejemplo para justificar su respuesta.
    \item Si se tira de los extremos de una cuerda en equilibrio con fuerzas de igual magnitud, pero dirección opuesta, ¿por qué la tensión total en la cuerda no es cero?
    \item ¿Existe alguna diferencia entre peso y masa? De un ejemplo para explicar su respuesta.
\end{enumerate}

\subsection*{Serie 2 (6pts)}
\textbf{Instrucciones: Desarrolle el siguiente problema, dejando constancia de sus procedimientos.}

\vspace{0.4cm}

Una nave de descenso se aproxima a la superficie de Calisto, uno de los satélites (lunas) del planeta Júpiter. Si el motor de la nave proporciona un empuje hacia arriba de $3260N$, la nave desciende a velocidad constante. Calisto no tiene atmósfera. Si el empuje hacia arriba es de $2200N$, la nave acelera hacia abajo a razón de $0.390m/s^2$. (a) ¿Cuál es el peso de la nave en descenso en la vecindad de la superficie de Calisto? (b) ¿Cuál es la masa de la nave? (c) ¿Cuál es la aceleración debido a la gravedad cerca de la superfice de Calisto?