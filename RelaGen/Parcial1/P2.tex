\section{Problema 2}
\subsection{}

Sea $s$ la trayectoria con parámetros de curva $\lambda$ y $\alpha$ tenemos que:

\begin{align*}
    s^{\mu}&=\left(t(\lambda,\alpha),x(\lambda,\alpha)\right)\\
    &= \left(\alpha\sinh\lambda,\alpha\cosh\lambda\right)
\end{align*}

Podemos hallar las líneas tangentes a esta curva derivando parcialmente respecto al parámetro, por tanto:

\textbf{Cuando $\mathbf{\alpha}$ es constante:}

\begin{align*}
    \pdv{s^\mu}{\lambda}&=\dv{s^\mu}{\lambda}\\
    &= \left(\alpha \cosh\lambda,\alpha\sinh\lambda\right)\\
    &= \chi^\mu
\end{align*}

\textbf{Cuando $\mathbf{\lambda}$ es constante:}

\begin{align*}
    \pdv{s^\mu}{\alpha}&=\dv{s^\mu}{\alpha}\\
    &=\left(\sinh\lambda,\cosh\lambda\right)\\
    &=\psi^\mu
\end{align*}

Ahora, para saber si estas líneas son ortogonales entre sí debemos calcular su producto punto:

\begin{align*}
    \chi^\mu\psi_\mu=\chi_\mu\psi^\mu&=\eta_{\mu\nu}\chi^\mu\psi^\nu\\
    &=\alpha\cosh\lambda\sinh\lambda-\alpha\sinh\lambda\cosh\lambda\\
    &=0
\end{align*}
\qed

\subsection{}

Calculemos $v$:
\begin{align*}
    v = \dv{x}{t}=\frac{\alpha\sinh\lambda}{\alpha\cosh\lambda}=\tanh\lambda
\end{align*}

Calculemos $\gamma$:
\begin{align*}
    \gamma = \frac{1}{\sqrt{1-v^2}} = \frac{1}{\sqrt{1-\tanh^2\lambda}} = \cosh\lambda
\end{align*}

Entonces:

\begin{align*}
    \dd \tau = \frac{\dd t}{\gamma} = \frac{\dd t}{\cosh\lambda}
\end{align*}

Y por la conservación de $\dd s^2$ tenemos:
\begin{align*}
    \dd s^2=\dd \tau^2 &= \dd t^2 - \dd x^2\\
    \dd \tau &= \dd t \left(1-\frac{\dd x^2}{\dd t^2}\right)^{\frac{1}{2}}\\
    \dd \tau &= \dd t \left(1-\cancelto{\tanh^2\lambda}{\left(\frac{\dd x}{\dd t}\right)^2}\right)^{\frac{1}{2}}\\
    \dd \tau &= \dd t \left[1-\tanh^2\lambda\right]^{\frac{1}{2}}\\
    \dd \tau &= \frac{\dd t}{\cosh\lambda}
\end{align*}



\subsection{}

Dados $\chi^\mu$ y $\psi^\mu$ del inciso 1, ahora podemos hallar los vectores normales $\hat{e}_\alpha$ y $\hat{e}_\lambda$ normalizando los hallados anteriormente.

\textbf{Con $\mathbf{\alpha}$ constante:}

\begin{align*}
    \hat{e}_\alpha &= \frac{1}{\sqrt{\chi^\mu\chi_\mu}}\chi^\mu\\
    &= \frac{1}{\sqrt{\eta_{\mu\nu}\chi^\mu\chi^\nu}}\chi^\mu\\\\
    \eta_{\mu\nu}\chi^\mu\chi^\nu &= \alpha^2(\cosh^2\lambda-\sinh^2\lambda)\\
    &=\alpha^2\\\\
    \hat{e}_\alpha^\mu = \frac{\chi^\mu}{\alpha}
\end{align*}

\textbf{Con $\mathbf{\lambda}$ constante:}

\begin{align*}
    \hat{e}_\lambda &= \frac{1}{\sqrt{\eta_{\mu\nu}\psi^{\mu}\psi^{\nu}}}\psi^\mu\\\\
    \eta_{\mu\nu}\psi^{\mu}\psi^{\nu}&=\sinh^2\lambda - \cosh^2\lambda\\
    &=-1\\\\
    \hat{e}_\lambda^\mu &= -i\psi^\mu
\end{align*}


Ahora hallemos en términos de los vectores base $\hat{e}_t=(1,0)$ y $\hat{e}_x=(0,-1)$ la nueva base:

\begin{align*}
    \begin{cases}
        \hat{e}_\alpha^\mu = \cosh\lambda\hat{e}_t^\mu-\sinh\lambda\hat{e}_x^\mu\\
        \hat{e}_\lambda^\mu = i\sinh\lambda\hat{e}_t^\mu-i\cosh\lambda\hat{e}_x^\mu
    \end{cases}
\end{align*}

Es decir:

\begin{align*}
    \begin{pmatrix}
        \hat{e}_\alpha^\mu \\ \hat{e}_\lambda^\mu
    \end{pmatrix} = 
    \begin{pmatrix}
        \cosh\lambda & \sinh\lambda\\
        i\sinh\lambda & i\cosh\lambda
    \end{pmatrix}
    \begin{pmatrix}
        \hat{e}_t^\mu\\\hat{e}_x^\mu
    \end{pmatrix}
\end{align*}

La matriz de transformación es:

\begin{align*}
    T = 
    \begin{pmatrix}
        \cosh\lambda & -\sinh\lambda\\
        i\sinh\lambda & -i\cosh\lambda
    \end{pmatrix}
\end{align*}


Y 

\begin{align*}
    T^{T} = 
    \begin{pmatrix}
        \cosh\lambda & i\sinh\lambda\\
        -\sinh\lambda & -i\cosh\lambda
    \end{pmatrix}
\end{align*}


Finalmente:

\begin{align*}
    TT^{T} =
    \begin{pmatrix}
        1   &  2i\sinh\lambda\cosh\lambda \\
        2i\sinh\lambda\cosh\lambda    & 1
    \end{pmatrix} =
    \begin{pmatrix}
        1   &  i\sinh2\lambda \\
        i\sinh2\lambda    & 1
    \end{pmatrix}
\end{align*}

No es ortogonal.

\subsection{}
Podemos hallar las geodésicas a partir de:

\begin{align*}
    \dd \tau^2 &= -\eta_{\alpha\beta}\dd x^\alpha\dd x^\beta\\
    &=\frac{\dd t^2}{\cosh^2\lambda}
\end{align*}

Tambien buscamos hallar:

\begin{align*}
    {\frac  {d^{2}x^{\mu}}{d\tau^{2}}}+\Gamma _{{\gamma\alpha}}^{{\mu}}{\frac  {dx^{\gamma}}{d\tau}}{\frac  {dx^{\alpha}}{d\tau}}=0
\end{align*}

Para esto haremos uso de las definiciones de cuadriaceleración y cuadrivelocidad, estos vectores son calculados en el siguiente inciso así que aquí únicamente las dejaré enunciadas:

\begin{align*}
    \alpha^\mu(\tau)+\Gamma_{\gamma\alpha}^\mu U^\gamma U^\alpha = 0
\end{align*}

\subsection{}

Tenemos el cuadrivector posición (que en este caso es bidimencional):
\begin{align*}
    x^\mu&=(ct,x,0,0)
    &=(c\alpha\sinh\lambda,\alpha\cosh\lambda,0,0)
\end{align*}

Ahora hallemos lo solicitado

\textbf{Cuadrivelocidad}

\begin{align*}
    U^\mu(\tau)&=\dv{x^\mu(\tau)}{\tau}\\
    &=\cosh\lambda\dv{x^\mu}{t}\\
    &=\cosh\lambda(1,v,0,0)\\
    &=\cosh\lambda(1,\tanh\lambda,0,0)
\end{align*}

\textbf{Cuadriaceleración}

\begin{align*}
    \alpha^\mu(\tau)&=\dv{U^\mu(\tau)}{\tau}\\
    % &=\cosh\lambda\dv{U^\mu}{t}\\
    &=\cosh\lambda\left(0,\dv{\tanh\frac{\tau}{\alpha}}{\tau},0,0\right)\\
    &=\cosh\lambda\left(0,\frac{1}{\alpha\cosh^2\frac{\lambda}{\alpha}},0,0\right)
\end{align*}

Además, vemos que $t$ y $x$ cumplen con la ecuación de una hiperbola:

\begin{align*}
    \frac{x^2}{\alpha^2}-\frac{t^2}{\alpha^2}=1
\end{align*}

Esto es fácil de demostrar, sustituyendo las definiciones de $x$ y $t$:

\begin{align*}
    \frac{x^2}{\alpha^2}-\frac{t^2}{\alpha^2}
        &=\frac{\alpha^2\cosh^2\lambda}{\alpha^2}-\frac{\alpha^2\sinh^2\lambda}{\alpha^2}\\
        &=\cosh^2\lambda-\sinh^2\lambda\\
        &=1
\end{align*}
\qed


\subsection{}

El la trayectoria descrita por el observador \textbf{O} sea una hiperbola se puede utilizar para hacer una relación al movimiento de partículas en el espacio sometidas a campos gravitacionales. Pero el que la matriz de transformación entre este sistema y los vectores de la base no sea ortogonal supondría dificultades de cálculo.

Cuando se tenga el caso específico en el que la trayectoria descrita coincida con el sistema, puede ser factible usar estos nuevos vectores base.